\documentclass[letterpaper,preprint,aps,pra,superscriptaddress]{revtex4-1}
\usepackage[T1]{fontenc}
\usepackage[latin9]{inputenc}
\usepackage{units}
\usepackage{amsmath}
\usepackage{amssymb}
\usepackage{graphicx}
\usepackage{wasysym}
\usepackage{braket}

\def\kr{k_{\rm R}}                            				% kr
\def\Er{E_{\rm R}}                            				% kr

\def\ex{\mathbf{e}_x}  
\def\es{\mathbf{e}_s}  
\def\ey{\mathbf{e}_y}  
\def\ez{\mathbf{e}_z}  

\usepackage{babel}
\begin{document}

\title{Diffraction based DMD wavemeter}

\author{I. B. Spielman}
\affiliation{Joint Quantum Institute, National Institute of Standards and Technology,
and University of Maryland, Gaithersburg, Maryland, 20899, USA}

\date{\today}
\begin{abstract}
Here I have some notes describing the physics of a wavemeter based on a DMD.
\end{abstract}
\maketitle

\section{Basic operation}


\begin{acknowledgments}
This work was partially supported by ...
\end{acknowledgments}

\bibliographystyle{apsrev4-1}
% \bibliography{StripePhase}

\end{document}
